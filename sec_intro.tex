Heavy-ion collisions at RHIC and the LHC create hot, dense matter whose space-time evolution is well described by relativistic viscous hydrodynamics~\cite{Gale:2013da, Heinz:2013th}. Owing to strong event-by-event density fluctuations in the initial state, the distributions of the final-state particles also fluctuate event-by-event. These fluctuations lead to harmonic modulation of the particle densities in the azimuthal angle $\phi$, characterized by a Fourier expansion $\text{d}N/\text{d}\phi\propto 1+2\sum v_n\text{cos}n(\phi-\Phi_n)$, where $v_n$ and $\Phi_n$ represent the magnitude and event-plane angle of the $n^\text{th}$-order harmonic flow. These quantities can also be conveniently represented by the per-particle "flow vector" $\pmb{v}_n=v_ne^{-\text{i}n\Phi_n}$ in each event. The measurements o harmonic flow coefficients $\pmb{v}_n$, and their event-by-event fluctuations, have places important constraints on the properties of the medium and on the density fluctuations in the initial state.

One important observable for studying event-by-event fluctuations of the initial condition as well as the final state dynamics of the medium is $p(v_n)$, the probability density distribution of the $v_n$, for event selected with similar centrality. The $p(v_n)$ are directly related to event-by-event fluctuations of the eccentricity $\epsilon_n$ associated with the $n^\text{th}$-order shape component in the initial state, $p(\epsilon_n)$~\cite{Gardim:2011xv, Gale:2012rq}. Measurement of $p(v_n)$ has been performed at the LHC for $n=2,3$ and 4 using an unfolding technique. While the $p(v_n)$ shape is approximately described by a Gaussian fluctuation of the underlying flow vector, significant deviations from Gaussian are observed at large $v_2$ in mid-central and peripheral collisions, and at large $v_3$ in mid-central collisions. Further detailed study of this non-Gaussian behavior, however, is limited by large uncertainties in the tail of the $p(v_n)$ distribution arising from the unfolding procedure.

An alternative way to study the $p(v_n)$ is through multi-particle azimuthal correlations within the cumulant framework~\cite{Borghini:2000sa, Bilandzic:2010jr}. This method calculates the quantities $c_n\{2k\}$, known as the $2k$-particle cumulants for the $n^\text{th}$-order flow harmonics. In the absence of non-flow correlations, such as resonance decay, jets etc, the $c_n\{2k\}$ are related to the moments of the $p(v_n)$, and therefore are sensitive to the shape of the $p(v_n)$ distribution. Most models of the initial state of A+A collisions predict a $p(v_n)$ whose shape is close to Gaussian, and that the four-particle cumulants $c_n\{4\}$ are zero or negative. The $c_n\{4\}$ for $n=2,3$ and 4 have been measured at RHIC and the LHC. The values of $c_2\{4\}$ and $c_3\{4\}$ are found to be negative, except for $c_2\{4\}$ in very central Au+Au collisions at RHIC where it is positive. The origin of this positive $c_2\{4\}$ is not understood. Furthermore, ATLAS has shown that $c_4\{4\}$ is negative in central collisions but becomes positive in mid-central and peripheral collisions. This sign change has been interpreted as nonlinear model-mixing effects between $v_4$ and $v_2$, i.e $v_4$ contains nonlinear contributions that are proportional to $v_2^2$.

In the cumulant framework, the $p(v_n,v_m)$ is studied using the so-called four-particle "symmetric cumulants", $sc_{n,m}\{4\}=\lr{v_n^2 v_m^2}-\lr{v_n^2}\lr{v_m^2}$~\cite{Bilandzic:2013kga} or the three-particle "asymmetric cumulants" such as $ac_{n,2n}\{3\}=\lr{v_n^2 v_{2n} \cos(2n(\Phi_n-\Phi_{2n}))}$~\cite{Jia:2017hbm}, which is sensitive to $p(v_n,v_m)$. The symmetric cumulants involve only the magnitude of the flow vector, while the asymmetric cumulants involve both the magnitude and phase of the flow vector. For this reason, asymmetric cumulants are often referred to as the "event-plane correlators". The $sc_{2,3}\{4\}$, $sc_{2,4}\{4\}$ and $ac_{2,4}\{3\}$ have been measured previously in A+A collisions~\cite{ALICE:2016kpq}. The values of $sc_{2,3}\{4\}$ are found to be negative, reflecting an anti-correlation between $v_2$ and $v_3$, while the positive values of $sc_{2,4}\{4\}$ and $ac_{2,4}\{3\}$ suggest a positive correlation between $v_2$ and $v_4$, consistent with the nonlinear mode-mixing effects mentioned before.

In heavy-ion collisions, $v_n$ coefficients are often calculated for events with similar activity, defined as the particle multiplicity in a fixed pseudorapidity range. Due to fluctuations in the particle production process, the centrality for events selected to have the same particle multiplicity fluctuates from event to event. Since the $v_n$ coefficients change with centrality, any fluctuation of centrality may lead to additional fluctuations of $v_n$, which broaden the underlying $p(v_n)$ and $p(v_n,v_m)$ distributions~\cite{Zhou:2018fxx}. Therefore the cumulants $c_n\{2k\}$, symmetric cumulants $sc_{n,m}\{4\}$ and asymmetric cumulants $ac_n\{3\}$ could be affected by the centrality resolution effects associated with a given event class definition. This centrality fluctuations, more commonly known as volume fluctuations, have been shown to contribute significantly to the event-by-event fluctuation of conserved quantities, especially in ultra-central collisions due to the steeply falling centrality distribution~\cite{Skokov:2012ds}. Recently, the volume fluctuations are found to also influence flow fluctuations, and is responsible for the sign-change of the $c_2\{4\}$ in ultra-central collisions~\cite{ATLAS-CONF-2017-066}. Therefore, a detailed study $c_n\{2k\}$, $sc_{n,m}\{4\}$ and $ac_{n,2n}\{3\}$ for different choices of reference event classes helps to clarify the meaning of centrality and provide insights on the sources for particle production in heavy-ion collisions. In this paper, two reference event classes are used in the calculation of cumulants to study the influence of volume fluctuations: the total transverse energy in the forward pseudorapidity $3.2<|\eta|<4.9$ and the number of reconstructed charged particles in mid-rapidity $|\eta|<2.5$.

One weakness of the standard multi-particle cumulant method is that it may not suppress adequately the non-flow correlation, which, although expected to be very small in A+A collisions, could in principle be responsible for the positive $c_n\{4\}$ values discussed above. These non-flow correlations can be further suppressed using a three-subevent cumulant method, which is based on correlation of particles from three different subevents separated in pseudorapidity $\eta$. This three-subevent method has successfully used in $pp$ and $p$+Pb collisions to suppress non-flow correlations~\cite{Aaboud:2017blb}, which are much stronger than in A+A collisions, to obtain a $c_n\{4\}$ associated with long-range collective flow. Therefore, the influence of non-flow on $c_n\{4\}$ in the standard method can be quantified by comparing with the three-subevent method.

This internal note presents a measurement of cumulant $c_n\{2k\}$, symmetric cumulant $sc_{n,m}\{4\}$ and asymmetric cumulant $ac_{n,2n}\{3\}$ in Pb+Pb collisions at $\sqrt{s_\text{NN}}=5.02$ TeV with the ATLAS detector. The results are obtained with the standard cumulant method, as well as with the three-subevent cumulant method to quantify the influence of non-flow correlations. The highlights of this measurement include: 1) first study of $c_1\{4\}$ associated with $\epsilon_1$, the dipole fluctuations in the initial state, 2) investigation of the centrality and $p_\text{T}$ dependence of $c_2\{4\}$ and $c_3\{4\}$, especially the sign of these quantities in ultra-central collisions, 3) precision study of $c_4\{4\}$ to understand the role of mode-mixing effects, 4) quantify the flow fluctuation using model-independent observable and 5) study of the correlation among different flow harmonics through symmetric cumulant and asymmetric cumulant. This internal note is organized as follows.
\begin{itemize}
\item Section.~\ref{sec:pre} summarizes the previous published results that are related with this analysis;
\item Section.~\ref{sec:evtSel} describes the ATLAS detector, trigger, and offline event selections;
\item Section.~\ref{sec:trkSel} contains a description of selection criteria for charged-particle tracks, as well as details of the Monte Carlo simulation samples used to derive the tracking efficiency and fake-track rates;
\item Section.~\ref{sec:method} outlined the detailed cumulant method;
\item Section.~\ref{sec:ana} outlined the detailed analysis procedure;
\item Section.~\ref{sec:sys} contains detailed discussions of the systematic errors;
\item Section.~\ref{sec:result} presented all the results of this analysis;
\item Section.~\ref{sec:paper} summarizes the plots to be included in the paper;
\item Section.~\ref{sec:summary} is devoted to summary and conclusions;
\item Section.~\ref{sec:appendix} contains comprehensive summary of systematics and results;
\end{itemize}



