\subsection{Outline}
The details of cumulant analysis are carried out in the following procedures:
\begin{itemize}
\item Calculation of 2-, 4- and 6-particle correlation $corr_n\{2k\}$:
\begin{itemize}
\item Standard cumulant method;
\item 3-subevent cumulant method;
\end{itemize}
\item Calculation of 2-, 4- and 6-particle cumulant $c_n\{2k\}$:
\begin{itemize}
\item Standard cumulant method;
\item 3-subevent cumulant method;
\end{itemize}
\item Calculation of 2-, 4- and 6-particle flow signal $v_n\{2k\}$;
\item Calculation of normalized cumulant $nc_n\{2k\}$;
\item Universality check of flow fluctuation models;
\item Calculation of symmetric cumulant $sc_{n,m}\{4\}$ and $nsc_{n,m}\{4\}$;
\begin{itemize}
\item Standard symmetric cumulant method;
\item 3-subevent symmetric cumulant method;
\end{itemize}
\item Calculation of asymmetric cumulant $ac_{n,n+m}\{3\}$ and $nac_{n,n+m}\{3\}$;
\begin{itemize}
\item Standard asymmetric cumulant method;
\item 3-subevent asymmetric cumulant method;
\end{itemize}
\end{itemize}



\subsection{Calculation of 2-, 4- and 6-particle correlation $corr_n\{2k\}$}
2-, 4- and 6-particle correlations are defined as:
\begin{equation}
\begin{split}
corr_n\{2\}&\equiv \lr{e^{\text{i}n(\phi_i-\phi_j)}} \\
corr_n\{4\}&\equiv \lr{e^{\text{i}n(\phi_i+\phi_j-\phi_k-\phi_l)}} \\
corr_n\{6\}&\equiv \lr{e^{\text{i}n(\phi_i+\phi_j+\phi_k-\phi_l-\phi_m-\phi_n)}}
\end{split}
\end{equation}
where notation $corr_n\{2k\}$ is used for the $2k$-particle correlations and $n$ denotes harmonic $n$ in the Fourier coefficients $v_n$. $i, j, k, l, m, n$ denotes unique particles in certain phase space $(p_\text{T},\eta)$, which will be quantified in the analysis section. $\lr{...}$ is the weighted average calculated for each event
\begin{equation}
\begin{split}
\lr{e^{\text{i}n(\phi_i-\phi_j)}}&\equiv \frac{\sum^{'}{w_i w_j e^{\text{i}n(\phi_i-\phi_j)}}}{\sum^{'}{w_i w_j}} \\
\lr{e^{\text{i}n(\phi_i+\phi_j-\phi_k-\phi_l)}}&\equiv \frac{\sum^{'}{w_i w_j w_k w_l e^{\text{i}n(\phi_i+\phi_j-\phi_k-\phi_l)}}}{\sum^{'}{w_i w_j w_k w_l}} \\
\lr{e^{\text{i}n(\phi_i+\phi_j+\phi_k-\phi_l-\phi_m-\phi_n)}}&\equiv \frac{\sum^{'}{w_i w_j w_k w_l w_m w_n e^{\text{i}n(\phi_i+\phi_j+\phi_k-\phi_l-\phi_m-\phi_n)}}}{\sum^{'}{w_i w_j w_k w_l w_m w_n}}
\end{split}
\end{equation}
where $\sum^{'}$ means the summation of unique particles: i.e. $i\neq j$, $i\neq j\neq k\neq l$ and $i\neq j\neq k\neq l\neq m\neq n$ respectively. $w$ is the weight applied to each particle, which is a combination of tracking efficiency $\epsilon$, fraction of fake tracks $f$ and trigger re-weighting $w_{trig}$:
\begin{equation}
w\equiv\frac{w_{\phi}(1-f)}{\epsilon}
\end{equation}
where all these weights will be discussed in details in the cumulant analysis section ~\ref{sec:ana}.

The most straightforward way to calculate $2k$-particle correlation $corr_n\{2k\}$ is called nested loop method: counting all the possible unique combinations within $2k$ nested loops of tracks. Since nested loop method has a complexity of $\mathcal{O}(M^{2k})$, where $M$ is the multiplicity in each event, it requires a lot of CPU hours to compute the 6-particle correlation, especially in Pb+Pb collision. An equivalent way is named as $Q$-cumulant (or direct-cumulant) method, which calculates $corr_n\{2k\}$ in a single loop, thus greatly reduces the complexity to $\mathcal{O}(M)$. The $Q$-cumulant method carefully removes all the correlations between same particles ("duplicates") by using simple diagrams. In this note, we will only list all the formula using $Q$-cumulant, without going into details about the derivation, and we have confirmed that both nested loop and $Q$-cumulant methods give identical results, which validates all the formulas we have used for the Q-cumulant method.


\subsubsection{Standard $Q$-cumulant method}
The event-by-event $\pmb{Q}_{n,k}$ vector in standard cumulant method is defined as:
\begin{equation}
\pmb{Q}_{n,k}\equiv\sum{w_i^k e^{\text{i}n\phi_i}}
\end{equation}
where $w_i$ is the particle weight introduced earlier and the power $k$ is for the purpose of removing duplicates. $n$ denotes the harmonic $n$ from the Fourier coefficients $v_n$.

In order to simply the expression, $S_{p,k}$ is introduced as:
\begin{equation}
S_{p,k}\equiv(\sum{w_i^k})^{p}
\end{equation}
where $k$ in $S_{p,k}$ is the same one with $k$ in $\pmb{Q}_{n,k}$. Note that unlike $\pmb{Q}_{n,k}$, $S_{p,k}$ is not related to the azimuthal angle $\phi$ of each particle.

The $\pmb{Q}_{n,k}$ and $S_{p,k}$ are defined in this way so that $2k$-particle correlation $corr_n\{2k\}$ can be expressed as a function of $\pmb{Q}_{n,k}$ and $S_{p,k}$:
\begin{equation}
corr_n\{2k\}=f(\pmb{Q}_{n,k},S_{p,k})
\end{equation}

The event-by-event 2-, 4- and 6-particle correlations in the standard $Q$-cumulant method can then be written as~\cite{Bilandzic:2010jr}:
\begin{equation}
\begin{split}
corr_n\{2\}&=\frac{|Q_{n,1}|^2-S_{1,2}}{S_{2,1}-S_{1,2}} \\
corr_n\{4\}&=\frac{|Q_{n,1}|^4+|Q_{2n,2}|^2-2\mathcal{R}\textit{e}(\pmb{Q}_{2n,2}\pmb{Q}_{n,1}^*\pmb{Q}_{n,1}^*)+8\mathcal{R}\textit{e}(\pmb{Q}_{n,3}\pmb{Q}_{n,1}^*)-4S_{1,2}|Q_{n,1}|^2+2S_{2,2}-6S_{1,4}}{S_{4,1}+8S_{1,3}S_{1,1}-6S_{1,2}S_{2,1}+3S_{2,2}-6S_{1,4}} \\
corr_n\{6\}&=(|Q_{n,1}|^6-6|Q_{n,1}|^2\mathcal{R}\textit{e}(\pmb{Q}_{2n,2}\pmb{Q}_{n,1}^*\pmb{Q}_{n,1}^*)+9|Q_{2n,2}|^2|Q_{n,1}|^2+4\mathcal{R}\textit{e}(\pmb{Q}_{3n,3}\pmb{Q}_{n,1}^*\pmb{Q}_{n,1}^*\pmb{Q}_{n,1}^*) \\
&+18S_{1,2}\mathcal{R}\textit{e}(\pmb{Q}_{2n,2}\pmb{Q}_{n,1}^*\pmb{Q}_{n,1}^*)-36\mathcal{R}\textit{e}(\pmb{Q}_{2n,4}\pmb{Q}_{n,1}^*\pmb{Q}_{n,1}^*)-36\mathcal{R}\textit{e}(\pmb{Q}_{n,3}\pmb{Q}_{n,1}\pmb{Q}_{2n,2}^*)+18S_{2,2}|Q_{n,1}|^2 \\
&-54S_{1,4}|Q_{n,1}|^2-72S_{1,2}\mathcal{R}\textit{e}(\pmb{Q}_{n,3}\pmb{Q}_{n,1}^*)+36|Q_{n,3}|^2+144\mathcal{R}\textit{e}(\pmb{Q}_{n,5}\pmb{Q}_{n,1}^*)-9S_{1,2}|Q_{n,1}|^4 \\
&+36|Q_{n,1}|^2\mathcal{R}\textit{e}(\pmb{Q}_{n,3}\pmb{Q}_{n,1}^*)-9S_{1,2}|Q_{2n,2}|^2+36\mathcal{R}\textit{e}(\pmb{Q}_{2n,4}\pmb{Q}_{2n,2}^*)-12\mathcal{R}\textit{e}(\pmb{Q}_{3n,3}\pmb{Q}_{2n,2}^*\pmb{Q}_{n,1}^*) \\
&+4|Q_{3n,3}|^2+54S_{1,4}S_{1,2}-6S_{3,2}-120S_{1,6})/(S_{6,1}-15S_{1,2}S_{4,1}+40S_{1,3}S_{3,1}+45S_{2,2}S_{2,1} \\
&-90S_{1,4}S_{2,1}-120S_{1,3}S_{1,2}S_{1,1}-15S_{3,2}+144S_{1,5}S_{1,1}+90S_{1,4}S_{1,2}+40S_{2,3}-120S_{1,6})
\end{split}
\end{equation}


\subsubsection{3-subevent $Q$-cumulant method}
Compared with standard method, the format of $corr_n\{2k\}$ are slightly altered in the 3-subevent method:
\begin{equation}
\begin{split}
corr_n^{a|b}\{2\}&\equiv \lr{e^{\text{i}n(\phi_a-\phi_b)}} \\
corr_n^{a,a|b,c}\{4\}&\equiv \lr{e^{\text{i}n(\phi_a+\phi_a^{'}-\phi_b-\phi_c)}}
\end{split}
\end{equation}
where notation $a|b$ and $a,a|b,c$ are added to the superscript of $corr_n\{2k\}$ to distinguish formula from the standard method. Moreover, particles $i,j,k$ and $l$ come from 3 subevents with different $\eta$ ranges:
\begin{itemize}
\item $\phi_a$: $\phi$ angle of particle from subevent a;
\item $\phi_b$: $\phi$ angle of particle from subevent b;
\item $\phi_c$: $\phi$ angle of particle from subevent c;
\item $\phi_a^{'}$: $\phi$ angle of particle from subevent a, but different from $\phi_a$;
\end{itemize}
One of the main reasons to measure cumulant is to suppress non-flow contribution, which originates from resonance decay, HBT, jet correlation and so on. In contrast to flow, non-flow usually has fewer particles associated with. By measuring the multi-particle correlation, those non-flow contributions can be mostly removed. However, for example, in the jet scenario, more than 3 particles can be correlated with each other, which can not be removed using the standard cumulant method. Due to this reason, subevent cumulant method is introduced to suppress the residual non-flow. In 3 subevent method, since the 4 particles are required to come from 3 subevents across the whole $\eta$ range, short-range (in $\eta$) non-flow correlations are greatly suppressed. Furthermore, 3-subevent is also robust at reducing long-range non-flow correlations, i.e. back-to-back di-jet correlation. The two correlated jets can only fall into two out of the three subevents, thus there will always be at least one particle in $corr_n\{4\}$ that is not associated with the di-jet. After averaging all the combinations, the di-jet correlation is significantly suppressed. The residual di-jet contribution can be easily evaluated by introducing small $\eta$ gaps between 3 subevents.

The subevent method has been extensively studied and validated in Monte-Carlo models as well as $pp$ data, where the non-flow contribution is much larger than Pb+Pb. The whole purpose of showing subevent results is to confirm that the non-flow is negligible in Pb+Pb: the final results will be presented using standard cumulant method after showing it gives same results as subevent method, since one advantage of standard method is that it has smaller statistical uncertainties. Due to the same reason, 6-particle cumulant is only calculated using standard method, as the fraction of non-flow that containing 6 or more particles is significantly lower.

Due to symmetry, there are other five ways to construct $corr_n\{4\}$ in 3-subevent:
\begin{equation}
\begin{split}
corr_n^{b,b|c,a}\{4\}&\equiv \lr{e^{\text{i}n(\phi_b+\phi_b^{'}-\phi_c-\phi_a)}} \\
corr_n^{c,c|a,b}\{4\}&\equiv \lr{e^{\text{i}n(\phi_c+\phi_c^{'}-\phi_a-\phi_b)}} \\
corr_n^{a,b|a,c}\{4\}&\equiv \lr{e^{\text{i}n(\phi_a+\phi_b-\phi_a^{'}-\phi_c)}} \\
corr_n^{b,c|b,a}\{4\}&\equiv \lr{e^{\text{i}n(\phi_b+\phi_c-\phi_b^{'}-\phi_a)}} \\
corr_n^{c,a|c,b}\{4\}&\equiv \lr{e^{\text{i}n(\phi_c+\phi_a-\phi_c^{'}-\phi_b)}}
\end{split}
\end{equation}
where the first two cases are simply permutations on the default configuration of four particles: 2 particles can come from either subevent a, b or c; These two cases are independent from the default and together all three cases will be included in the 3-subevent cumulant calculation of this analysis. We will briefly discuss how to merge the $corr_n\{4\}$ from these three cases later. While for the last three cases, since terms like $\phi_a-\phi_a^{'}$ calculates the correlation within one subevent, which contains much larger fraction of short-range non-flow contribution. So in this analysis, we will not include the last three cases.

Compared with standard cumulant method, since the number of duplicates in summation $\sum$ are significantly less, the formula of $corr_n\{2k\}$ for 3-subevent is much simpler:
\begin{equation}
\begin{split}
corr_n^{a|b}\{2\}&=\frac{\mathcal{R}\textit{e}(\pmb{Q}_{n,1}^{a}\pmb{Q}_{n,1}^{b*})}{S_{1,1}^a S_{1,1}^b} \\
corr_n^{a,a|b,c}\{4\}&=\frac{\mathcal{R}\textit{e}(\pmb{Q}_{n,1}^a \pmb{Q}_{n,1}^{b*} \pmb{Q}_{n,1}^a \pmb{Q}_{n,1}^{c*})-\mathcal{R}\textit{e}(\pmb{Q}_{2n,2}^a \pmb{Q}_{n,1}^{b*} \pmb{Q}_{n,1}^{c*})}{(S_{2,1}^a-S_{1,2}^a)S_{1,1}^b S_{1,1}^c}
\end{split}
\end{equation}
The other similar configurations $corr_n^{b|c}\{2\}$, $corr_n^{c|a}\{2\}$, $corr_n^{b,b|c,a}\{4\}$ and $corr_n^{c,c|a,b}\{4\}$ can be easily written by permutations of the indices $a$, $b$ and $c$.


\subsection{Calculation of 2-, 4- and 6-particle cumulant $c_n\{2k\}$}
In the last section, event-by-event $2k$-particle correlation $corr_n\{2k\}$ has been calculated. In this section, we will calculate the $2k$-particle cumulant by combining $corr_n\{2k\}$ with different orders.

Cumulant is defined on the ensemble of similar events, noted as "event class". The average of $corr_n\{2k\}$ in each event class is defined as:
\begin{equation}
\lr{corr_n\{2k\}}\equiv\frac{\sum W_i\{2k\}corr_n\{2k\}}{\sum W_i\{2k\}}
\end{equation}
where the summation $\sum$ is over every event in one event class. $W_i\{2k\}$ is the number of unique multiplets in each event, which will be defined later. Event weight from trigger prescale $w_{trig}$ should also be multiplied to $W_{i}\{2k\}$ (see Sec.\ref{sec:ana}). Since cumulant measures the flow fluctuation within one event class, how the event class is defined could change the magnitude, even the sign of cumulants. For the analysis the default event class definition is centrality, with $1\%$ as the bin width. But in the analysis section we will dive into more variations of the definitions of event class.

$2k$-particle cumulant is defined as a combination of 2-, 4- ... $2k$-particle correlation:
\begin{equation}
c_n\{2k\}\equiv f(corr_n\{2\},corr_n\{4\}, ... ,corr_n\{2k\})
\end{equation}
where all the lower order terms $corr_n\{2\}$, $corr_n\{4\}$, ... ,$corr_n\{2k-2\}$ are used to remove the lower-order-particle correlation from $2k$-particle correlation and the final remaining $c_n\{2k\}$ is referred as "genuine" particle correlation, which corresponds to flow or collectivity. With the idea of subevent introduced, the formula of subevent cumulant will also slightly change compared with standard cumulant, which will be discussed in details in the following sections.


\subsubsection{Standard cumulant}
Without particle weights, the event weight is simply defined as number of combinations in an event:
\begin{equation}
\begin{split}
W\{2\}&\equiv M(M-1) \\
W\{4\}&\equiv M(M-1)(M-2)(M-3) \\
W\{6\}&\equiv M(M-1)(M-2)(M-3)(M-4)(M-5)
\end{split}
\end{equation}
where $M$ is the multiplicity in each event. With particle weights, the formula are more complicated due to the duplicates (correlation among same particles):
\begin{equation}
\begin{split}
W\{2\}&\equiv S_{2,1}-S_{1,2} \\
W\{4\}&\equiv S_{4,1}+8S_{1,3}S_{1,1}-6S_{1,2}S_{2,1}+3S_{2,2}-6S_{1,4} \\
W\{6\}&\equiv S_{6,1}-15S_{1,2}S_{4,1}+40S_{1,3}S_{3,1}+45S_{2,2}S_{2,1}-90S_{1,4}S_{2,1}-120S_{1,3}S_{1,2}S_{1,1}-15S_{3,2} \\
&+144S_{1,5}S_{1,1}+90S_{1,4}S_{1,2}+40S_{2,3}-120S_{1,6}
\end{split}
\end{equation}
where $S_{p,k}$ is defined in earlier sections. Note that the event weights are also the denominators of the $2k$-particle correlations.

Finally, 2-, 4- and 6-particle cumulants are defined as:
\begin{equation}
\begin{split}
c_n\{2\}&= \lr{corr_n\{2\}}; \\
c_n\{4\}&= \lr{corr_n\{4\}}-2\lr{corr_n\{2\}}^2; \\
c_n\{6\}&= \lr{corr_n\{6\}}-9\lr{corr_n\{4\}}\lr{corr_n\{2\}}+12\lr{corr_n\{2\}}^3;
\end{split}
\end{equation}


\subsubsection{3-subevent cumulant}
Without particle weights, the event weight is simply defined as number of combinations in an event:
\begin{equation}
\begin{split}
W^{a|b}\{2\}&\equiv M_{a}M_{b} \\
W^{a,a|b,c}\{4\}&\equiv M_{a}(M_{a}-1)M_{b}M_{c}
\end{split}
\end{equation}
where $M_{a}$ and $M_{b}$ are multiplicity in subevent $a$ and $b$ repectively. Superscripts $a|b$ and $a,a|b,c$ are used to label the configurations of 3-subevents, and there are other two configurations for $W\{4\}$: $W^{b,b|c,a}\{4\}$ and $W^{c,c|a,b}\{4\}$, which can be easily derived by permutation of the indices $a$, $b$ and $c$.
Similarly, the event weights with particle weights are:
\begin{equation}
\begin{split}
W^{a|b}\{2\}&\equiv S_{1,1}^a S_{1,1}^b \\
W^{a,a|b,c}\{4\}&\equiv (S_{2,1}^a-S_{1,2}^a)S_{1,1}^b S_{1,1}^c
\end{split}
\end{equation}

2- and 4-particle cumulants are defined as:
\begin{equation}
\begin{split}
c_n^{a|b}\{2\}&= \lr{corr_n^{a|b}\{2\}}; \\
c_n^{a,a|b,c}\{4\}&= \lr{corr_n^{a,a|b,c}\{4\}}-2\lr{corr_n^{a|b}\{2\}}\lr{corr_n^{a|c}\{2\}}
\end{split}
\end{equation}

Once $c_n^{a,a|b,c}\{4\}$, $c_n^{b,b|c,a}\{4\}$ and $c_n^{c,c|a,b}\{4\}$ are calculated, they will be combined, weighted by the corresponding event weight, to make the final $c_n^{3-sub}\{4\}$ in 3-subevent cumulant method. This will triple the total statistical of 3-subevent method, since all the three configurations are statistically independent.
\begin{equation}
c_n^{\text{3-sub}}\{4\} \equiv \frac{(\sum W^{a,a|b,c}\{4\}) c_n^{a,a|b,c}\{4\} + (\sum W^{b,b|c,a}\{4\}) c_n^{b,b|c,a}\{4\} + (\sum W^{c,c|a,b}\{4\}) c_n^{c,c|a,b}\{4\}}{\sum W^{a,a|b,c}\{4\} + \sum W^{b,b|c,a}\{4\} + \sum W^{c,c|a,b}\{4\}}
\end{equation}
Due to event-by-event multiplicity fluctuation along $\eta$, $\sum W^{a,a|b,c}\{4\}$, $\sum W^{b,b|c,a}\{4\}$ and $\sum W^{c,c|a,b}\{4\}$ will not be same with each other. This will cause different statistical significance among $c_n^{a,a|b,c}\{4\}$, $c_n^{b,b|c,a}\{4\}$ and $c_n^{c,c|a,b}\{4\}$. Due to this reason, the total summation are weighted by the corresponding total event weight.



\subsection{Calculation of 2-, 4- and 6-particle flow signal $v_n\{2k\}$}
Before converting the cumulant to flow signal, in order to increase the statistical significance, the cumulant are re-binned to larger bin width, e.g. from $1\%$ to $10\%$ centrality, weighted by the total number of events in each event class $N_{evt}$:
\begin{equation}
c_n^{rebin}\{4\}\equiv \frac{\sum_{i\in \text{event classes}} (N_{evt}^i)c_n^i\{4\}}{N_{evt}^i}
\end{equation}

Different order cumulants provide independent estimates for the same reference harmonic $v_n$. If the underlying $v_n$ fluctuation is Bessel-Gaussian or close to Bessel-Gaussian (e.g. power-law function), then the $2k$-particle cumulant can be expanded as:
\begin{equation}
\begin{split}
c_n\{2\} &= \bar{v}_n^2+2\delta_n^2 \\
c_n\{4\} &= -\bar{v}_n^4 \\
c_n\{6\} &= \bar{v}_n^6
\end{split}
\end{equation}
where $\bar{v}_n$ denotes the mean value of $v_n$ and $\delta_n$ describes the Gaussian fluctuation width of $v_n$. Thus flow signal $v_n\{2k\}$ for the corresponding cumulant $c_n\{2k\}$ can be defined as:
\begin{equation}
\begin{split}
v_n\{2\} &= \sqrt{c_n\{2\}} \\
v_n\{4\} &= \sqrt[4]{-c_n\{4\}} \\
v_n\{6\} &= \sqrt[6]{\frac{1}{4}c_n\{6\}}
\end{split}
\end{equation}
where above equations are universal for standard and subevent cumulant methods.



\subsection{Calculation of normalized cumulant $nc_n\{2k\}$}
Multi-particle cumulant not only is affected by the flow fluctuation, but also changes with the mean value of flow $\bar{v}_n$. So the centrality and $p_\text{T}$ dependence of cumulant partially originated from the centrality and $p_\text{T}$ dependence of $\bar{v}_n$. In order to disentangle the flow fluctuation from the mean value of flow, an observable was previously defined to show relative flow fluctuation:
\begin{equation}
\sqrt{\frac{v_n^2\{2\}-v_n^2\{4\}}{v_n^2\{2\}+v_n^2\{4\}}}=\frac{\sigma_v}{\bar{v}}
\end{equation}
where $\sigma_v$ reflects the fluctuation width of the event-by-event $v_n$. However, one main defect of this observable is that in order to obtain the R.H.S. of the formula, one has to assume the underlying flow fluctuation is Gaussian. As will be seen in this analysis, this assumption is not true, especially in peripheral and central collision.

Instead, a simpler observable is defined that is related to the cumulant ratios~\cite{Sirunyan:2017fts}, and it's notated as the "normalized cumulant":
\begin{equation}
\begin{split}
nc_n\{4\}&\equiv\frac{c_n\{4\}}{c_n^2\{2\}} = (\frac{v_n\{4\}}{v_n\{2\}})^4 \\
nc_n\{6\}&\equiv\frac{c_n\{6\}}{4c_n^3\{2\}} = (\frac{v_n\{6\}}{v_n\{2\}})^6
\end{split}
\end{equation}
where a factor of 4 in the definition of $nc_n\{6\}$ is to properly remove the normalization factor in the definition of $c_n\{6\}$, so that the normalized cumulant falls into the range of $(-1,1)$. This observable is named as the normalized cumulant since if follows the definition of normalized symmetric cumulant, where similar normalization terms are applied to the symmetric cumulant. The centrality and $p_\text{T}$ dependence of $c_n\{2k\}$ partially originates from those dependence of $\lr{v_n^2}$, so that after normalization, $nc_n\{2k\}$ mainly reflects the fluctuation itself. Another advantage of using normalized cumulant is for the plotting purpose: there is no longer need to zoom in the Y-axis when the $c_{n}\{2k\}$ is too small.



\subsection{Universality check of flow fluctuation models}
Following the previous discussions, if the flow fluctuation is Gaussian, the 4- and higher even-order particle cumulants should result in the same flow signal $v_{n}\{2k\}$. In this section, we will quantify the flow fluctuation and compare two competing models.~\cite{Yan:2013laa}

For the eccentricity $\epsilon_{n}$ in the initial stage, based on previous theoretical studies~\cite{Yan:2013laa}, there are two major fluctuation models on the market: Gaussian and power-law. If the eccentricity fluctuation is Gaussian, then the cumulants of eccentricity can be calculated explicitly:
\begin{equation}
\begin{split}
\epsilon_{n}\{2\}&= \sqrt{\bar{\epsilon}^{2}+\sigma^{2}} \\
\epsilon_{n}\{4\}&= \bar{\epsilon} \\
\epsilon_{n}\{6\}&= \bar{\epsilon}
\end{split}
\end{equation}
where $\bar{\epsilon}$ is the mean value of the eccentricity and $\sigma^{2}$ is the variance. Similarly, if the eccentricity fluctuation is power-law, then the cumulant of eccentricity can also be calculated explicitly:
\begin{equation}
\begin{split}
\epsilon_{n}\{2\}&= \sqrt{\frac{1}{1+\alpha}} \\
\epsilon_{n}\{4\}&= \sqrt[4]{\frac{2}{(1+\alpha)^2(2+\alpha)}} \\
\epsilon_{n}\{6\}&= \sqrt[6]{\frac{6}{(1+\alpha)^3(2+\alpha)(3+\alpha)}}
\end{split}
\end{equation}
where $\alpha$ is the single parameter in the power-law function.

In the hydro-dynamical picture, the eccentricity $\epsilon_{n}$ in the initial stage and flow $v_{n}$ are linearly correlated:
\begin{equation}
v_{n}\{2k\}=\kappa_{n}\epsilon_{n}\{2k\}
\end{equation}
where $\kappa_{n}$ is the scaling factor and depends on the harmonic $n$.

In order to derive a universality check for the Gaussian fluctuation, both $\epsilon$ and $\kappa_{n}$ need to be canceled out, then we get:
\begin{equation}
\frac{c_n\{6\}}{4(-c_n\{4\})^{\frac{3}{2}}}=1
\end{equation}
where $c_n\{4\}$ and $c_n\{6\}$ are 4- and 6-particle cumulants respectively. By universality, it means that if the flow fluctuation is Gaussian, then the quantity on the L.H.S. should be equal to 1. Any derivation from 1 will indicate that the fluctuation is away from Gaussian. Similarly, the universality check for power-law fluctuation can also be calculated:
\begin{equation}
\frac{c_n\{6\}(2c_n^2\{2\}-c_n\{4\})}{12c_n\{2\}c_n^2\{4\}}=1
\end{equation}
where 2-particle cumulant $c_n\{2\}$ is also included. Note that the inclusion of 2-particle cumulant might introduce non-flow, but for the universality check, we are only focusing central and mid-central collisions, where the non-flow contributions are minimal compared with flow. The results of both checks will be compared in the measurement section.



\subsection{Calculation of symmetric cumulant $sc_{n,m}\{4\}$ and $nsc_{n,m}\{4\}$}
\subsubsection{Standard symmetric cumulant}
The symmetric cumulant measures the correlation and fluctuation between harmonics $v_n$ and $v_m$ $(n<m)$. The 4-particle correlation with mixed harmonics is defined as:
\begin{equation}
corr_{n,m}\{4\}\equiv\lr{e^{\text{i}(n\phi_i+m\phi_j-n\phi_k-m\phi_l)}}
\end{equation}
where $n$ and $m$ denote the order of harmonics $v_n$ and $v_m$. $"\lr{}"$ is the event-by-event mean value weighted by the particle weight. Similarly, we can apply the Q-cumulant technique to calculate $corr_{n,m}\{4\}$ in a single loop~\cite{Jia:2017hbm}:
\begin{equation}
\begin{split}
corr_{n,m}\{4\}&=(|Q_{n,1}|^2|Q_{m,1}|^2-2\mathcal{R}\textit{e}(\pmb{Q}_{n+m,2}\pmb{Q}_{n,1}^*\pmb{Q}_{m,1}^*)-2\mathcal{R}\textit{e}(\pmb{Q}_{m-n,2}\pmb{Q}_{n,1}\pmb{Q}_{m,1}^*)+|Q_{n+m,2}|^2 \\
&+|Q_{m-n,2}|^2+4\mathcal{R}\textit{e}(\pmb{Q}_{n,3}\pmb{Q}_{n,1}^*)+4\mathcal{R}\textit{e}(\pmb{Q}_{m,3}\pmb{Q}_{m,1}^*)-S_{1,2}|Q_{n,1}|^2-S_{1,2}|Q_{m,1}|^2 \\
&+S_{2,2}-6S_{1,4})/(S_{4,1}+8S_{1,3}-6S_{1,2}S_{2,1}+3S_{2,2}-6S_{1,4})
\end{split}
\end{equation}
where note that the denominator is same as the denominator of the 4-particle correlation $corr_n\{4\}$.

Then the event-by-event $corr_{n,m}\{4\}$ is averaged within each event class, with the weight $W\{4\}$:
\begin{equation}
W\{4\}\equiv S_{4,1}+8S_{1,3}-6S_{1,2}S_{2,1}+3S_{2,2}-6S_{1,4}
\end{equation}

Finally, the symmetric cumulant, $sc_{n,m}\{4\}$, is defined as:
\begin{equation}
sc_{n,m}\{4\} = \lr{corr_{n,m}\{4\}}-\lr{corr_n\{2\}}\lr{corr_m\{2\}}
\end{equation}
where $corr_n\{2\}$ is simply the 2-particle correlation calculated in the cumulant section.

One caveat of symmetric cumulant is that it not only reflects the correlation between $v_n$ and $v_m$, but also is scaled by the magnitudes of $v_n$ and $v_m$. To show the correlation part only, normalized cumulant, $nsc_{n,m}\{4\}$, is defined as:
\begin{equation}
nsc_{n,m}\{4\} = \frac{sc_{n,m}\{4\}}{\lr{v_n^2}\lr{v_m^2}}
\end{equation}
where by dividing the flow magnitudes of $v_{n}$ and $v_{m}$, only correlation remains in the $nsc_{n,m}\{4\}$. $\lr{v_n^2}$ denotes the mean value of 2-particle $v_n$, which has been calculated by the previous 2-particle correlation. In order to reduce the non-flow in the estimate of $\lr{v_n^2}$, we will use the calculated 3-subevent $c_n^{b|c}\{2\}$, with an $\eta$ gap between subevent $b$ and $c$:
\begin{equation}
\begin{split}
\lr{v_n^2} &= c_n^{b|c}\{2\} \\
\lr{v_m^2} &= c_m^{b|c}\{2\}
\end{split}
\end{equation}

\subsubsection{3-subevent symmetric cumulant}
Similarly, 4-particle correlation with mixed harmonics can be defined in 3-subevent:
\begin{equation}
corr_{n,m}^{a,a|b,c}\{4\}\equiv\lr{e^{\text{i}(n\phi_i+m\phi_j-n\phi_k-m\phi_l)}}
\end{equation}
where the superscript $a,a|b,c$ represents the subevent that particles $i,j,k,l$ come from:
\begin{itemize}
\item particles $i$ and $j$ come from subevent $a$;
\item particle $k$ comes from subevent $b$;
\item particle $l$ comes from subevent $c$;
\end{itemize}

There are 12 different unique permutation for $corr_{n,m}^{a,a|b,c}\{4\}$, where 6 of them have small non-flow since two particles that come from the same subevent have the same sign in $n\phi_i+m\phi_j-n\phi_k-m\phi_l$. In this analysis, we will calculate the following 6 configurations then combine them on the cumulant level:
\begin{itemize}
\item $corr_{n,m}^{a,a|b,c}\{4\}$
\item $corr_{n,m}^{a,a|c,b}\{4\}$
\item $corr_{n,m}^{b,b|c,a}\{4\}$
\item $corr_{n,m}^{b,b|a,c}\{4\}$
\item $corr_{n,m}^{c,c|a,b}\{4\}$
\item $corr_{n,m}^{c,c|b,a}\{4\}$
\end{itemize}
and for simplicity, we will only list the formula for the first case. The formula for other cases can be derived easily by permutations.

After applying the Q-cumulant technique, $corr_{n,m}^{a,a|b,c}\{4\}$ can be calculated in a single loop:
\begin{equation}
corr_{n,m}^{a,a|b,c}\{4\} = \frac{\mathcal{R}\textit{e}(\pmb{Q}_{n,1}^a\pmb{Q}_{m,1}^a\pmb{Q}_{n,1}^{b*}\pmb{Q}_{m,1}^{c*})-\mathcal{R}\textit{e}(\pmb{Q}_{n+m,2}^a\pmb{Q}_{n,1}^{b*}\pmb{Q}_{m,1}^{c*})}{(S_{2,1}^a-S_{1,2}^{a})S_{1,1}^b S_{1,1}^c}
\end{equation}
where all the variables are defined previously.

The symmetric cumulant using 3-subevent method, $sc_{n,m}^{3-sub}\{4\}$, is defined as:
\begin{equation}
sc_{n,m}^{3-sub}\{4\} = \lr{corr_{n,m}^{a,a|b,c}\{4\}} - \lr{corr_n^{a|b}\{2\}}\lr{corr_m^{a|c}\{2\}}
\end{equation}
where $corr_n^{a|b}\{2\}$ is the 2-particle correlation calculated in the 3-subevent cumulant section.

Similarly, the normalized symmetric cumulant using 3-subevent method, $nsc_{n,m}^{3-sub}\{4\}$, is defined as:
\begin{equation}
nsc_{n,m}^{3-sub}\{4\} = \frac{sc_{n,m}^{3-sub}\{4\}}{\lr{v_n^2}\lr{v_m^2}}
\end{equation}
where the denominator is calculated in the same way as normalized symmetric cumulant with standard method. Note that $\lr{v_n^2}$ are not calculated in adjacent subevents as $corr_n^{a|b}\{4\}$, otherwise non-flow will contribute to the normalization factors.



\subsection{Calculation of asymmetric cumulant $ac_{n,n+m}\{3\}$ and $nac_{n,n+m}\{3\}$}
\subsubsection{Standard asymmetric cumulant}
Symmetric cumulant measures the correlation between flow harmonics $v_n$ and $v_m$, to further evaluate the correlation among more harmonics $v_n$, $v_m$ and $v_{n+m}$, the asymmetric cumulant is proposed. 3-particle correlation with mixed harmonics is defined as:
\begin{equation}
corr_{n,m,n+m}\{3\}\equiv\lr{e^{\text{i}(n\phi_i+m\phi_j-(n+m)\phi_k)}}
\end{equation}
where note that the coefficient of the third particle $k$ need to be $n+m$ otherwise the mean value is 0. One advantage of asymmetric cumulant is that it only requires 3-particle correlation, which results in much better statistics than symmetric cumulant.

To calculate the 3-particle correlation in a single loop, formula with Q-cumulant technique is derived as~\cite{Jia:2017hbm}:
\begin{equation}
\begin{split}
corr_{n,m,n+m}\{3\} &= ( \mathcal{R}\textit{e}(\pmb{Q}_{n,1}\pmb{Q}_{m,1}\pmb{Q}_{n+m,1}^*)-\mathcal{R}\textit{e}(\pmb{Q}_{n+m,1}\pmb{Q}_{n+m,2}^*)-\mathcal{R}\textit{e}(\pmb{Q}_{n,1}\pmb{Q}_{n,2}^*) \\
&-\mathcal{R}\textit{e}(\pmb{Q}_{m,1}\pmb{Q}_{m,2}^*)+2S_{1,3} ) / (S_{3,1}-3S_{1,2}S_{1,1}+2S_{1,3})
\end{split}
\end{equation}
where all the variables are same as those in the cumulant section.

The event-by-event $corr_{n,m,n+m}\{3\}$ is then averaged within each event class, with the event weight $W\{3\}$:
\begin{equation}
W\{3\} = S_{3,1}-3S_{1,2}S_{1,1}+2S_{1,3}
\end{equation}

Finally, the asymmetric cumulant, $ac_{n,n+m}\{3\}$, is calculated:
\begin{equation}
ac_{n,n+m}\{3\} = \lr{corr_{n,m,n+m}\{3\}}
\end{equation}
where unlike cumulant or symmetric cumulant, the asymmetric cumulant is simply the average of 3-particle correlation with mixed harmonics. Like symmetric cumulant, in order to measure the pure correlation among $v_n$, $v_m$ and $v_{n+m}$, normalized asymmetric cumulant, $nac_{n,n+m}\{3\}$, is defined as:
\begin{equation}
nac_{n,n+m}\{3\} = \frac{ac_{n,n+m}\{3\}}{\sqrt{\lr{v_n^2 v_m^2}\lr{v_{n+m}^2}}}
\end{equation}
where $v_n^2$ denotes the 2-particle $v_n$. In the case where $n=m$:
\begin{equation}
\lr{v_n^2 v_m^2} = \lr{v_n^4}
\end{equation}
where $\lr{v_n^4}$ is related to the 4-particle cumulant, after non-flow is suppression. Using 3-subevent method:
\begin{equation}
\lr{v_n^4} = c_n^{3-sub}\{4\} + 2 (c_n^{3-sub}\{2\})^2
\end{equation}
In this analysis, since only $nac_{2,4}\{3\}$ is measured, there is no need to evaluate $\lr{v_n^2 v_m^2}$ separately: it can be calculated by reusing 3-subevent 2- and 4-particle cumulant results.


\subsubsection{3-subevent asymmetric cumulant}
In a similar way, 3-particle with mixed harmonics, using 3-subevent method, is defined as:
\begin{equation}
corr_{n,m,n+m}^{a,b|c}\{3\}\equiv\lr{e^{\text{i}(n\phi_i+m\phi_j-(n+m)\phi_k)}}
\end{equation}
where the superscript $a,b|c$ represents the subevent that particles $i,j,k$ come from:
\begin{itemize}
\item particle $i$ comes from subevent $a$;
\item particle $k$ comes from subevent $b$;
\item particle $l$ comes from subevent $c$;
\end{itemize}

There are 6 different unique permutation for $corr_{n,m}^{a,b|c}\{3\}$, which reduced to 3 unique cases in the case $n=m$. In this analysis, we will calculate the following 3 configurations then combine them on the cumulnat level:
\begin{itemize}
\item $corr_{n,n,2n}^{a,b|c}\{3\}$
\item $corr_{n,n,2n}^{b,c|a}\{3\}$
\item $corr_{n,n,2n}^{c,a|b}\{3\}$
\end{itemize}
and for simplicity, we will only list the formula for the first case. The formula for other cases can be derived easily by permutations.

After applying the Q-cumulant technique, event-by-event $corr_{n,m,n+m}^{a,b|c}$ can be calculated in a single loop:
\begin{equation}
corr_{n,m,n+m}^{a,b|c}\{3\} = \frac{\mathcal{R}\textit{e}(\pmb{Q}_{n,1}^a\pmb{Q}_{m,1}^b\pmb{Q}_{n+m,1}^{c*})}{S_{1,1}^a S_{1,1}^b S_{1,1}^c}
\end{equation}
where all the variables are defined previously.

The asymmetric cumulant using 3-subevent method, $ac_{n,n+m}^{3-sub}\{3\}$, is defined as:
\begin{equation}
ac_{n,n+m}^{3-sub}\{3\} = \lr{corr_{n,m,n+m}^{a,b|c}\{3\}}
\end{equation}

Similarly, the normalized asymmetric cumulant using 3-subevent method, $nac_{n,n+m}^{3-sub}\{3\}$, is defined as:
\begin{equation}
nac_{n,n+m}^{3-sub}\{3\} = \frac{ac_{n,n+m}^{3-sub}\{3\}}{\sqrt{\lr{v_n^2 v_m^2}\lr{v_{n+m}^2}}}
\end{equation}
where the denominator is calculated in the same way as normalized asymmetric cumulant using standard method.










