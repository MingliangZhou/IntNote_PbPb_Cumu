This internal note presents details of measurements of the multi-particle azimuthal anisotropy in lead-lead collisions at 5.02 TeV at the LHC in 2015. The measurements are performed for charged particles with various transverse momenta, from $0.5<p_{\text{T}}<5.0$ GeV to $2.0<p_{\text{T}}<5.0$ GeV, and in the pseudorapidity range $|\eta|<2.5$. The anisotropy is characterized by the cumulant form of Fourier coefficients, $c_{n}\{4\}$, of the charged-particle azimuthal angle distribution. The Fourier coefficients are evaluated using multi-particle cumulant calculated with the direct cumulant (Q-cumulant) method.

The features of this analysis is listed as follows:
\begin{itemize}
\item Cumulants are measured in different $p_\text{T}$ ranges, where we require all particles coming from the same high $p_\text{T}$ range. In previous measurements only one particle is from high $p_\text{T}$, which requires the assumption that cumulant is factorizable in $p_\text{T}$;
\item For the first time, cumulant of the dipolar flow $c_1\{4\}$ is found to be non-zero with particles from high $p_\text{T}$ range and the magnitude of $c_1\{4\}$ increases towards peripheral collisions;
\item We performed a precise measurement of higher-order harmonic $c_4$, as a function of centrality. $c_4\{4\}$ is found to be negative only in the central and mid-central collisions, which indicates an interplay between linear and non-linear contribution in the hydrodynamic picture;
\item With ample statistics from ultra-central collision triggers, detailed studies of cumulant in most-central collisions show interesting sign change behavior of $c_2\{4\}$ and $c_2\{6\}$, which could be explained by the volume fluctuation and centrality resolution;
\item We have tested two fluctuation models: Gaussian and power-law, and $v_2$ fluctuation is found to be closer to Gaussian than power-law, while the $v_3$ fluctuation is consistent with purely randomly fluctuation;
\item To study the correlation between different flow harmonics, symmetric and asymmetric cumulants are measured with high precision;
\item For all the observables above, the measurements are repeated with three different event class definitions. The purpose is to evaluate the potential centrality fluctuation effects on cumulant measurements, and we do observe differences in most of the observables;
\end{itemize}
